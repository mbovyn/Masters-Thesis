\chapter{Background} \label{ch:background}

The advent of video microscopy brought the dynamic nature of the organelle environment into focus in the early 1980s \cite{Allen1981,Inoue1981}. Soon after, Ron Vale and colleagues discovered the identity of one of the proteins responsible for the observed transport, kinesin \cite{Vale1985}. In the more than 30 years since this discovery, we have learned a great deal about each of the components of the transport system from a combination of careful in vivo and in vitro experimental work, as well as enlightening modeling efforts.

\section{Microtubules}

Eukaryotic cells possess a microtubule (MT) network, composed of hundreds of individual MT filaments. Recently, super-resolution microscopy has allowed imaging and mapping of the entire MT network with a single MT resolution \cite{Zhang2016}, as shown in figure \ref{fig:MTcytoskeleton}. Individual MTs are dynamic, assembling and disassembling stochastically \cite{Howard2003}. This allows the MT network to be reorganized as cells move and change shape, as well as in response to external cues \cite{Herms2015,Zhu2015,Zhang2016}. 

\begin{figure}
\centering
\includegraphics[scale=.5]{background/MTcytoskeleton}
\caption[Super-resolution image of a MT network]{An example of a super-resolution image of an entire MT network of a cell (A), adapted from \cite{Zhang2016}. Single MTs are identifiable from fluorescence labelling in red. Region in yellow box magnified in (B), region in green box magnified in (C). Scale bars are \SI{5}{\micro\meter} in panel A and \SI{2}{\micro\meter} in panels B and C.}
\label{fig:MTcytoskeleton}
\end{figure}

The MT network serves many functions in the cell. Most importantly for the work in this document, MTs serve as the ``roads'' along which cargos are transported. In addition to this role, MTs are a component of the of the cytoskeleton, which gives cells shape and structure. Furthermore, MTs bear the forces exerted to divide chromosomes between the daughter cells in cell division. 

The structure of MTs allow them to perform these functions. MTs are tubular, consisting of a number of linear protofilaments assembled into a ring \cite{Grimstone1966} as shown in figure \ref{fig:MTstructure}. MTs commonly have 13 or 14 protofilaments \cite{Pierson1978}, but that number can vary from fewer than 9 to more than 15 depending on organism and cell type\cite{Davis1983}. Each protofilament is a repeating polymer of dimers of $\alpha$ and $\beta$ tubulin subunits. This repeating structure makes it possible for kinesin and dynein to move processivley along the MT outer surface. MT structures can have defects, and it has been shown these defects can influence transport \cite{Liang2016}.

\begin{figure}
\floatbox[{\capbeside\thisfloatsetup{capbesideposition={right,top}}}]{figure}[\FBwidth]
{\caption[Microtubule Structure]{The structure of a microtubule is shown, demonstrating the tubular structure composed of linear protofilaments made up of repeating dimers of $\alpha$ and $\beta$ subunits. Illustration \href{https://en.wikipedia.org/wiki/Microtubule\#/media/File:Microtubule_structure.png}{``Structure of a microtubule''} by \href{https://commons.wikimedia.org/wiki/User:Splette}{Thomas Splettstoesser} is licensed under \href{https://creativecommons.org/licenses/by-sa/4.0/}{CC BY-SA 4.0.}}
\label{fig:MTstructure}}
{\includegraphics[scale=.1]{background/Microtubule_structure}}
\end{figure}

The tubular structure makes MTs rigid, with persistence lengths near \SI{1}{\milli\meter}, many times the size of an entire cell. Rigidity can vary with MT length and repeated bending \cite{Schaedel2015}.

\section{Molecular Motors}

Molecular motors are a class of enzymes (specifically ATPases), which convert the chemical energy of ATP into mechanical work. There are three classes of molecular motors involved in cargo transport: kinesin superfamiliy motors and cytoplasmic dynein which walk along microtubules, and myosin superfamily motors which walk along actin. Here we'll focus on kinesin.

\subsection{Kinesin} \label{sec:kinesin}

The defining feature of a kinesin is the kinesin motor domain, which binds to MTs and is responsible for ATP hydrolysis \cite{Verhey2011}. Hundreds of genes which include the kinesin motor domain have been identified across a variety of different organisms; together these genes make up the kinesin superfamily. This superfamily has been divided into 14 families based on function and sequence similarity \cite{Lawrence2004}. In mice, there are 45 individual kinesins which perform a variety of tasks in the cell, from transporting cargo to MT depolymerization \cite{Hirokawa2009}. The motors which have established roles in transport are the members of the kinesin-1 (KIF5), kinesin-2 (KIF3) and kinesin-3 (KIF1) families \cite{Verhey2011}. Kinesin-1 is the most well studied family, also called conventional kinesin. It is a heterotetramer of a kinesin heavy chain homodimer and two kinesin light chain subunits, as shown in figure \ref{fig:kinesin_types}. Kinesin-2 can exist either as a homodimer or heterodimer of different motor domain containing subunits, with each form associated with different motility properties. However, the motility mechanism of kinesin-2 is similar to kinesin-1 \cite{Andreasson2015b}. Kinesin-2 motors also tend to switch protofilaments while walking, resulting in a spiral path around the MT \cite{Brunnbauer2012}, where kinesin-1 motors walk a straight line \cite{Ray1993}. Kinesin-3 motors exhibit limited motility as a monomer, but likely work as dimers in vivo, similar to kinesin-1 and kinesin-2 motors \cite{Siddiqui2017}. We will focus on kinesin-1 for the rest of this section and will refer to it simply as kinesin.

\begin{figure}
\includegraphics[width=\textwidth]{background/kinesin_types}
\caption[Composition of transport kinesins]{Subunit composition of transport kinesins. Kinesin-1 or conventional kinesin is the most well studied motor family. Kinesin-2 and kinesin-3 are less studied, but also known to transport cargos in cells. Kinesin motor domins are the defining feature of the kinesin superfamily and are shown as green ovals. Kinesin motors domains bind to the MT, which the other end of the motors bind cargos or linkers. Abbreviations: FHA, forkhead associated; KAP, kinesin-associated protein; KHC, kinesin heavy chain; KIF, kinesin family; KLC, kinesin light chain; PH, pleckstrin homology; TPR, tetratricopeptide repeat. Figure from \cite{Verhey2011}.}
\label{fig:kinesin_types}
\end{figure}

Kinesin motors step processively along MT tracks in a hand-over-hand fashion, with each motor domain taking \SI{16}{\nano\meter} steps \cite{Yildiz2004} that move the center of mass of the motor forward by \SI{8}{\nano\meter} \cite{Svoboda1993}. When unloaded, kinesin steps quickly, moving about \SI{1}{\micro\meter\per\second}. Times between steps are exponentially distributed \cite{Carter2005}. Velocity reduces under increasing resistive load until stall, which occurs at $\approx$ \SI{6}{\pico\newton}. The nature of this decrease has been found to be superlinear in some studies \cite{Kunwar2010,Visscher1999,Fallesen2011,Rai2013}, while others find more linear behavior \cite{Svoboda1994,Andreasson2015a}.

Kinesin unbinds from the microtubule at a rate that also depends on the load experienced by the motor. Several reports agree that unbinding rate increases with load exponentially up to the stall force, but disagree about behavior above stall. One study claims unbinding rate increases only slowly above stall \cite{Kunwar2011}, while another claims unbinding rate continues to increase exponentially \cite{Andreasson2015a}. The unbinding rate has been found to depend on the directionality of the load applied. Hindering loads result in the aforementioned behavior, while assisting loads result in an increased unbinding rate \cite{Milic2014,Andreasson2015a}. Sideways loads result in slightly asymmetrical unbinding behavior \cite{Block2003}.

It has been found that the kinesin stalk is stiff in tension \cite{Kojima1997} and also resists compression to a lesser degree \cite{Jeney2004}. It has little resistance to torsion \cite{Hunt1993,Gutierrez-Medina2009}.

\section{Cargos}

A wide variety of organelle cargos such as lipid droplets, mitochondria, melanosomes, peroxisomes, pigment granules, endosomes, secretory vesicles, RNA granules, and virions are transported bi-directionally by molecular motors \cite{Hancock2014,Gross2004}. Long distance cargo transport is also particularly important in neurons. Cargos such as dense core vesicles carrying neuropeptide and synaptic vesicles carrying neurotransmitter must be transported down the length of axons, often millimeters to centimeters long, and delivered at presynapses \cite{Maeder2014}. These observations lead to a picture of cargo transport where bidirectional motion is controlled to correctly localize cargos, as shown in figure \ref{fig:cargo_delivery}.

\begin{figure}
\centering
\includegraphics[width=.45 \textwidth]{background/bidirectional_motion}
\includegraphics[width=.45 \textwidth]{background/cargo_delivery}
\caption[Bidirectional motion and cargo delivery]{A variety of cargos have been shown to move bidirectionally on MTs in cells, as shown on the left. By biasing this bidirectional motion, the cell may be able to deliver cargos to specific locations in neurons and other cell types, as shown on the right. Figures on left and bottom right from \cite{Hancock2014}. Figure on top right from \cite{Gross2004}.}
\label{fig:cargo_delivery}
\end{figure}

\section{Previous models of cargo transport}

An early model of multiple motor cargo transport posed the problem as a Markov chain, with state transitions representing binding and unbinding of the motors \cite{Klumpp2005}. It was shown this model was able to reproduce a variety of the behaviors observed in vivo \cite{Muller2008}. Despite its success, it was difficult to interpret how experimentally observed parameters translated to model parameters, including how forces are shared between multiple motors and the stochastic nature of motor stepping.

A subsequent model represented motors as two points, with forces in each generated by the stretch in spring-like motor stalks \cite{Kunwar2008}. While only one dimensional spatially, this model was able to elucidate many details of how multiple motors might work together \cite{Kunwar2010}. In a surprising negative result, the Gross lab showed that neither model was able to match the behavior of cargos in vivo when supplied with realistic single motor behavior \cite{Kunwar2011}.

Three dimensional models of cargo transport have also since been constructed \cite{Korn2009,Erickson2011,Lombardo2017}. They have been used to investigate the impact of how motors are arranged on a cargo \cite{Erickson2011} and cargo switching between filiments \cite{Erickson2013}. Recently, a model of cargo transport included the ability for motors to diffuse in the cargo membrane in 3D \cite{Lombardo2017}.