\thesistitle{Cargo Navigation Across 3D Microtubule Intersections}

%"Dissertation" for PhD, "Thesis" for master's
\documenttitle{Dissertation}

\degreename{Doctor of Philosophy}

% Use the wording given in the official list of degrees awarded by UCI:
% http://www.rgs.uci.edu/grad/academic/degrees_offered.htm
\degreefield{Physics}

% Your name as it appears on official UCI records.
\authorname{Matthew Jacob Bovyn}

% Use the full name of each committee member.
\committeechair{Professor Jun Allard}
\othercommitteemembers
{
  Professor Steve Gross, Co-Advisor\\
  Professor Albert Siryaporn
}

\degreeyear{2017}

\copyrightdeclaration
{
  {\copyright} {\Degreeyear} \Authorname
}

% If you have previously published parts of your manuscript, you must list the
% copyright holders; see Section 3.2 of the UCI Thesis and Dissertation Manual.
% Otherwise, this section may be omitted.
% \prepublishedcopyrightdeclaration
% {
% 	Chapter 4 {\copyright} 2003 Springer-Verlag \\
% 	Portion of Chapter 5 {\copyright} 1999 John Wiley \& Sons, Inc. \\
% 	All other materials {\copyright} {\Degreeyear} \Authorname
% }

% The dedication page is optional
% (comment out to exclude).
\dedications
{
  (Optional dedication page)
  
  To ...
}

\acknowledgments
{
Foremost, I'd like to thank my advisors Jun Allard and Steve Gross for their constant support. I would also like to thank my lab mates in the Allard Lab (Lara Clemens, Abdon Iniguez, Kathryn Manakova, and Derek Bryant) and in the Gross Lab (Babu Reddy and Dail Chapman) for productive conversations and general help of all kinds. I would also like to thank the members of the UCI Center for Biological Systems for productive and enlightening conversations, as well as the training I needed to take on this project.

I was supported by NIH T32 EB009418-07 to Arthur Lander and Qing Nie, as well as NIH R01 GM123068 to Jun Allard and Steve Gross.
  
  You also need to acknowledge any publishers of your previous
  work who have given you permission to incorporate that work
  into your dissertation. See Section 3.2 of the UCI Thesis and
  Dissertation Manual.)
}


% Some custom commands for your list of publications and software.
\newcommand{\mypubentry}[3]{
  \begin{tabular*}{1\textwidth}{@{\extracolsep{\fill}}p{4.5in}r}
    \textbf{#1} & \textbf{#2} \\ 
    \multicolumn{2}{@{\extracolsep{\fill}}p{.95\textwidth}}{#3}\vspace{6pt} \\
  \end{tabular*}
}
\newcommand{\mysoftentry}[3]{
  \begin{tabular*}{1\textwidth}{@{\extracolsep{\fill}}lr}
    \textbf{#1} & \url{#2} \\
    \multicolumn{2}{@{\extracolsep{\fill}}p{.95\textwidth}}
    {\emph{#3}}\vspace{-6pt} \\
  \end{tabular*}
}

% Include, at minimum, a listing of your degrees and educational
% achievements with dates and the school where the degrees were
% earned. This should include the degree currently being
% attained. Other than that it's mostly up to you what to include here
% and how to format it, below is just an example.
%
% CV is required for PhD theses, but not Master's
% comment out to exclude
\curriculumvitae
{

\textbf{EDUCATION}
  
  \begin{tabular*}{1\textwidth}{@{\extracolsep{\fill}}lr}
    \textbf{Doctor of Philosophy in Computer Science} & \textbf{2012} \\
    \vspace{6pt}
    University name & \emph{City, State} \\
    \textbf{Bachelor of Science in Computational Sciences} & \textbf{2007} \\
    \vspace{6pt}
    Another university name & \emph{City, State} \\
  \end{tabular*}

\vspace{12pt}
\textbf{RESEARCH EXPERIENCE}

  \begin{tabular*}{1\textwidth}{@{\extracolsep{\fill}}lr}
    \textbf{Graduate Research Assistant} & \textbf{2007--2012} \\
    \vspace{6pt}
    University of California, Irvine & \emph{Irvine, California} \\
  \end{tabular*}

\vspace{12pt}
\textbf{TEACHING EXPERIENCE}

  \begin{tabular*}{1\textwidth}{@{\extracolsep{\fill}}lr}
    \textbf{Teaching Assistant} & \textbf{2009--2010} \\
    \vspace{6pt}
    University name & \emph{City, State} \\
  \end{tabular*}

\pagebreak

\textbf{REFEREED JOURNAL PUBLICATIONS}

  \mypubentry{Ground-breaking article}{2012}{Journal name}

\vspace{12pt}
\textbf{REFEREED CONFERENCE PUBLICATIONS}

  \mypubentry{Awesome paper}{Jun 2011}{Conference name}
  \mypubentry{Another awesome paper}{Aug 2012}{Conference name}

\vspace{12pt}
\textbf{SOFTWARE}

  \mysoftentry{Magical tool}{http://your.url.here/}
  {C++ algorithm that solves TSP in polynomial time.}

}

% The abstract should not be over 350 words, although that's
% supposedly somewhat of a soft constraint.
\thesisabstract
{
  Eukaryotic cells maintain a high level of internal organization and change it dynamically to respond to their environment. The localization of organelles such as mitochondria and lipid droplets is often determined by their transport along the microtubule cytoskeleton by molecular motors in the kinesin superfamily and cytoplasmic dynein. The single molecule function of these motors has been intensely studied, but it remains unclear how the motile properties of these molecules are harnessed by the cell to localize cargos to the correct place. The work in this document explores how the properties of the motors, the cellular environment, and the cargos themselves conspire direct intracellular traffic. First, we explore how the organization of a basic element of the cellular environment --- the microtubule cytoskeleton --- can determine cargo routing. In a second project, I propose to investigate how the manner in which motors are attached to the cargo can influence transport. In a third project, I propose a combination of experimental and computational studies to examine the relationship between the binding rate of individual motor proteins to microtubules and binding rates of the cargos they are attached to. Furthermore, I propose to investigate how modulation of this relationship may allow the cell to regulate transport.
  
   Active transport of intracellular cargos along microtubules is essential for organization and function of eukaryotic cells. The properties of the molecular motor proteins that power this transport have been intensely studied on the single molecule level, but it remains unclear how these motors function in groups and how the cell controls the function of the motors to deliver cargos to the correct place at the correct time. Furthermore, the cell must direct the transport of a variety of cargos with different destinations along the same network of microtubule ``roadways'', using the same set of molecular motors to power their transport. The projects presented here address mechanisms by which the cell might achieve this multiplexed control.

The projects investigate cargo transport through mathematical models and through experiments, both in vitro and in cells. The mathematical models are derived by using force balance to find the equations of motion for the cargo, modeled as a spherical rigid body. A Brownian force term is included, making the equations stochastic. To solve these equations, along with implementing motor stepping, binding and unbinding, we've written a hybrid Euler-Maryuama-Gillespie simulation. The experiments make use of differential interference contrast microscopy to visualize cells and organelles, particularly lipid droplets. Optical tweezers are used to measure forces on cargos created by molecular motors both in vitro and in cells.
  
  In collaboration with the Vershinin Lab at University of Utah, we investigate how cargos navigate microtubule intersections in 3D. While cargo transport is often studied only along a single microtubule, the microtubule network is usually a dense mesh, with microtubules frequently crossing each other close enough together for cargos to encounter both microtubules simultaneously. When cargos encounter such an intersection, do they switch to the intersecting microtubule or pass it by? The Vershinin Lab conducted a series of in vitro experiments with kinesin motors attached to glass bead cargos and found that the probability for the cargo to switch tracks depends strongly on the geometry of the intersection. We constructed a simulation to investigate how the properties of the molecular motors and cargo lead to the switching probabilities observed. We report that well known properties of the molecular motors lead to switching probabilities close to those observed in experiment. Furthermore, we report that observed switching probabilities stem from unexpected effects, such as cargo diffusion being constrained by motors bound to the microtubule. \textbf{Our results predict high switching rates for intersections in which microtubules are spaced less than one quarter of a cargo diameter apart, suggesting the cell could regulate switching by modulating microtubule spacing.} In this work, theory allows insight into mechanisms which would otherwise be unobservable (either because of experimental limitations, like finding the number of motors engaged, or because they are in general not directly observable, like force states). This insight allows us to predict what features of the experimental data are likely to generalize to other cargos and in vivo situations, and which are not. This work is currently in revision.
}


%%% Local Variables: ***
%%% mode: latex ***
%%% TeX-master: "thesis.tex" ***
%%% End: ***
