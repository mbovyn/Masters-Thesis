\thesistitle{Geometry Matters for Cargos Navigating 3D Microtubule Intersections}

%"Dissertation" for PhD, "Thesis" for master's
\documenttitle{Thesis}

\degreename{Master of Science}

% Use the wording given in the official list of degrees awarded by UCI:
% http://www.rgs.uci.edu/grad/academic/degrees_offered.htm
\degreefield{Chemical and Materials Physics - Physics}

% Your name as it appears on official UCI records.
\authorname{Matthew Jacob Bovyn}

% Use the full name of each committee member.
\committeechair{Associate Professor Jun Allard}
\othercommitteemembers
{
  Professor Steven Gross, Co-Advisor\\
  Assistant Professor Albert Siryaporn
}

\degreeyear{2018}

\copyrightdeclaration
{
  {\copyright} {\Degreeyear} \Authorname
}

% If you have previously published parts of your manuscript, you must list the
% copyright holders; see Section 3.2 of the UCI Thesis and Dissertation Manual.
% Otherwise, this section may be omitted.
% \prepublishedcopyrightdeclaration
% {
% 	Chapter 4 {\copyright} 2003 Springer-Verlag \\
% 	Portion of Chapter 5 {\copyright} 1999 John Wiley \& Sons, Inc. \\
% 	All other materials {\copyright} {\Degreeyear} \Authorname
% }

% The dedication page is optional
% (comment out to exclude).
%\dedications
%{
%  (Optional dedication page)
%  
%  To ...
%}

\acknowledgments
{
Foremost, I'd like to thank my advisors Jun Allard and Steve Gross for their constant support. I would also like to thank my lab mates in the Allard Lab (Lara Clemens, Abdon Iniguez, Kathryn Manakova, and Derek Bryant) and in the Gross Lab (Babu Reddy and Dail Chapman) for productive conversations and general help of all kinds. I would also like to thank the members of the UCI Center for Biological Systems for productive and enlightening conversations, as well as the training I needed to take on this project.

I was supported during the development of the model by NIH T32 training grant EB009418-07 to Arthur Lander and Qing Nie in the UCI Center for Complex Biological Systems. I am currently supported by NSF Integrative Graduate Education and Research Traineeship (IGERT) grant DGE-1144901 to Vasan Venugopalan in the Beckman Laser Center. I was also supported by NIH R01 GM123068 to Jun Allard and Steve Gross.
}


% Some custom commands for your list of publications and software.
\newcommand{\mypubentry}[3]{
  \begin{tabular*}{1\textwidth}{@{\extracolsep{\fill}}p{4.5in}r}
    \textbf{#1} & \textbf{#2} \\ 
    \multicolumn{2}{@{\extracolsep{\fill}}p{.95\textwidth}}{#3}\vspace{6pt} \\
  \end{tabular*}
}
\newcommand{\mysoftentry}[3]{
  \begin{tabular*}{1\textwidth}{@{\extracolsep{\fill}}lr}
    \textbf{#1} & \url{#2} \\
    \multicolumn{2}{@{\extracolsep{\fill}}p{.95\textwidth}}
    {\emph{#3}}\vspace{-6pt} \\
  \end{tabular*}
}

% Include, at minimum, a listing of your degrees and educational
% achievements with dates and the school where the degrees were
% earned. This should include the degree currently being
% attained. Other than that it's mostly up to you what to include here
% and how to format it, below is just an example.
%
% CV is required for PhD theses, but not Master's
% comment out to exclude
%\curriculumvitae
%{
%
%\textbf{EDUCATION}
%  
%  \begin{tabular*}{1\textwidth}{@{\extracolsep{\fill}}lr}
%    \textbf{Doctor of Philosophy in Computer Science} & \textbf{2012} \\
%    \vspace{6pt}
%    University name & \emph{City, State} \\
%    \textbf{Bachelor of Science in Computational Sciences} & \textbf{2007} \\
%    \vspace{6pt}
%    Another university name & \emph{City, State} \\
%  \end{tabular*}
%
%\vspace{12pt}
%\textbf{RESEARCH EXPERIENCE}
%
%  \begin{tabular*}{1\textwidth}{@{\extracolsep{\fill}}lr}
%    \textbf{Graduate Research Assistant} & \textbf{2007--2012} \\
%    \vspace{6pt}
%    University of California, Irvine & \emph{Irvine, California} \\
%  \end{tabular*}
%
%\vspace{12pt}
%\textbf{TEACHING EXPERIENCE}
%
%  \begin{tabular*}{1\textwidth}{@{\extracolsep{\fill}}lr}
%    \textbf{Teaching Assistant} & \textbf{2009--2010} \\
%    \vspace{6pt}
%    University name & \emph{City, State} \\
%  \end{tabular*}
%
%\pagebreak
%
%\textbf{REFEREED JOURNAL PUBLICATIONS}
%
%  \mypubentry{Ground-breaking article}{2012}{Journal name}
%
%\vspace{12pt}
%\textbf{REFEREED CONFERENCE PUBLICATIONS}
%
%  \mypubentry{Awesome paper}{Jun 2011}{Conference name}
%  \mypubentry{Another awesome paper}{Aug 2012}{Conference name}
%
%\vspace{12pt}
%\textbf{SOFTWARE}
%
%  \mysoftentry{Magical tool}{http://your.url.here/}
%  {C++ algorithm that solves TSP in polynomial time.}
%
%}

% The abstract should not be over 350 words, although that's
% supposedly somewhat of a soft constraint.
\thesisabstract
{
Eukaryotic cells transport cargos along microtubules to control their distribution within the cell and deliver them to distant locations. While we understand how molecular motors can transport cargos along individual microtubules, the cell's microtubules are usually arranged in a complex 3D network. While traversing this network, cargos need to navigate intersections where microtubules cross at a wide variety of separation distances and angles. To gain insight into how cargos navigate these intersections, we have used a recently established 3D construction technique based on holographic optical trapping to build single 3D microtubule intersections in vitro with relevant nanoscale precision. We then used these fully suspended microtubule structures to perform motility assays on kinesin-1 coated cargos. We find that some intersection geometries influence cargos to pass along their current microtubule, while other geometries influence them to switch to the intersecting one. To understand how, we use a 3D Brownian dynamics simulation of cargo transport to investigate the mechanisms which give rise to the observed switching probabilities. Using these stochastic simulations, we find that switching probability is often determined by a competition between a stronger motor team on the primary microtubule and the intersecting microtubule sterically hindering that team?s progress. This understanding of the basic mechanisms of switching at single intersections in 3D helps lay a foundation for understanding how the cell may regulate switching to control how cargos navigate the MT network and ultimately their spatial organization.
}


%%% Local Variables: ***
%%% mode: latex ***
%%% TeX-master: "thesis.tex" ***
%%% End: ***
