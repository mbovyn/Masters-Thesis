\chapter{Introduction}

Life on earth is mindblowingly diverse. For anything you might try to pin down as the defining feature --- be it DNA, metabolism, reproduction, or anything else you can dream of, counterexamples and edge cases spring up to defeat you. It is instructive to look instead not at individual elements, but at design principles: what are the common ways evolution has wrangled various arrangements of carbon, nitrogen, oxygen and other trace elements to build up bacteria, elephants, and jelly fish? (Not to mention the human that wrote the words on this page, and the one reading them.) One fundamental motif is the barrier; all life on earth uses barriers to contain the chemical reactions which give rise to the properties we associate with being alive: movement, metabolism, reproduction, and so on.

While bacteria have only a single major barrier which functions to separate their carefully controlled interior environment from the outside world, eukaryotic cells have in addition developed a multitude of internal barriers. To construct these internal barriers, eukaryotic cells mostly use lipid bilayers. The organelles encapsulated by these internal barriers allow eukaryotic cells to maintain a variety of vastly different chemical environments, which together can be used to accomplish complex tasks not achievable in any one environment alone. 

For this strategy to succeed in keeping the cell functioning, different enclosed environments need to communicate with one another. While communication is possible at a distance, it is often more effective --- and sometimes necessary --- for organelles to make physical contact with each other to transfer molecules or pass along signals. Controlling contact between organelles --- and more generally controlling cell spatial organization --- requires that organelles move around relative to each other. While thermal energy causes significant motion even on the scale of an organelle, the relatively weak and undirected motion is not sufficient for delivering cargos in a reasonable time. 

To this end, eukaryotic cells have built up a complex transport system to control the organization of their organelles in space and time. To transport organelles, eukaryotic cells have evolved to use molecular motors, which transform the cell's chemical energy currency, ATP, into mechanical work. These motors move along filamentous tracks: the molecular motors kinesin and dynein move along the large, rigid microtubules, while myosin motors move along the thinner, more floppy actin filaments.

The general question which naturally arises from the ideas so far established is this: how does the cell control the transport of various cargos from one place to another? There are many types of cargos, each of which with different functions which may necessitate them to be localized differently from each other at different times. How is intracellular traffic directed? This is the question we will concern ourselves with.
